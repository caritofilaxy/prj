\documentclass[oneside,final,14pt]{extreport}
\usepackage{graphicx}
\usepackage[T2A,T1]{fontenc}
\usepackage[utf8]{inputenc}
\usepackage[russian,english]{babel}
\usepackage{vmargin}
\setmarginsrb{2cm}{1.5cm}{1cm}{1.5cm}{0pt}{0mm}{0pt}{13mm}

\begin{document}


\title{Introduction to \LaTeX{}}
\author{Author's Name}

\maketitle

\begin{abstract}
The abstract text goes here.
\end{abstract}

\section{Introduction}
Here is the text of your introduction.

\begin{equation}
    \label{simple_equation}
    \alpha = \sqrt{ \beta }
\end{equation}

\subsection{Subsection Heading Here}
Write your subsection text here.

\begin{otherlanguage*}{russian}
\subsection{Слоны дут на север}
% Устаревшая форма декларирущих команд
У попа {\it была} {\bf собака}

% Декларирующие команды (scope)
\noindentВ этой \itshape фразе почти все \bfseries 
набрано курсивом, \ttfamily но кое что
\upshape еще и жирное, \rmfamily а где-то
и вовсе \mdseries моноширинное.

\noindentCначала {\bfseries жирное, потом {\itshape курсив, потом} без курсива, потом} без жирного.

% Команды с параметром
\noindentСначала \textbf{жирное, потом \textit{курсив, потом} без курсива, потом} без жирного.

% Команды в форме окружения
\noindentСначала \begin{bfseries} жирное, потом 
        \begin{itshape} курсив, потом
        \end{itshape}без курсива, потом
        \end{bfseries} без жирного.


\end{otherlanguage*}

\subsection{Subsection Heading Here next}
Write your subsection text here next.

\section{Conclusion}
Write your conclusion here.

\end{document}
